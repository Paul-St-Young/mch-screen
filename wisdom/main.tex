\documentclass[aps,prb,showpacs,preprintnumbers,amsmath,amssymb,superscriptaddress,twocolumn]{revtex4-1}
\usepackage{amsmath}
\usepackage{graphicx}
\usepackage{comment}
\usepackage{hyperref}
\usepackage{braket}
\usepackage{xcolor}
%\usepackage{CJK}
\usepackage{float}
\usepackage{enumitem}

\graphicspath{{./figures/}}

\newcommand{\bs}{\boldsymbol}
\newcommand{\up}{\uparrow}
\newcommand{\down}{\downarrow}

\newcommand{\ssec}[1]{\emph{#1} ---}

\definecolor{gold}{rgb}{0.83, 0.69, 0.22}
\newcommand{\paul}[1]{\textcolor{gold}{\textbf{YY: #1}}}

\newlist{todolist}{itemize}{2}
\setlist[todolist]{label=$\square$}

\begin{document}
%\begin{CJK*}{UTF8}{}
\title{Effect of gate screening in transition metal dichalcogenide heterobilayer}
\author{Wisdom Boinde}
\affiliation{Center for Computational Quantum Physics, Flatiron Institute, New York, NY, 10010, USA}
\author{Yubo Yang}
\affiliation{Center for Computational Quantum Physics, Flatiron Institute, New York, NY, 10010, USA}
\author{Miguel Morales}
\affiliation{Center for Computational Quantum Physics, Flatiron Institute, New York, NY, 10010, USA}
\date{\today}
\begin{abstract}
Metallic gates in transition metal dichalcogenide heterobilayer devices screen the long-range Coulumb interaction between charge carriers.
How does this change the phases/properties of the electronic states?
\end{abstract}
\pacs{}
\maketitle
%\end{CJK*}

\ssec{Introduction}
% background and motivation
TMDC bilayers are cool~\cite{Mak2022}.
Could screened TMDC bilayer, shown in Fig.~\ref{fig:device}, be cooler?
\begin{figure*}
\begin{minipage}{0.38\textwidth}
\includegraphics[width=\linewidth]{{unscreened}.png}
(a) unscreened
\end{minipage}
\begin{minipage}{0.48\textwidth}
\includegraphics[width=\linewidth]{{screened}.png}
(b) screened
\end{minipage}
\caption{Examples of TMDC bilayer experimental device. Metallic gates can be moved closed to the charge carriers to screen the interaction among them.}
\label{fig:device}
\end{figure*}

\ssec{Model}
% effective model for charge carriers
The moir\'e continuum hamiltonian (MCH)

\ssec{Methods}
Density functional theory (DFT)

\ssec{Summer Milestones (draft 05/30)}
\begin{itemize}
\item Learning targets for June 14
\begin{todolist}
\item Describe TMDC continuum model: the MCH
\item Compute band structure using Quantum Espresso (QE)
\item Compute charge density using QE
\item Solve the MCH (at one condition) using QE
\item Compute spin density using QE
\end{todolist}
\item Compute targets for June 28
\begin{todolist}
\item Map out MCH phase diagram (no screening)
\item Study charge density, spin density, and momentum distribution
\item Poster outline (intro, method, expected results)
\end{todolist}
\item Research targets for July 12
\begin{todolist}
\item Implement screening in QE
\item Investigate effect of screening
\end{todolist}
\item Production targets for July 26
\begin{todolist}
\item Map out MCH phase diagram with screening
\end{todolist}
\item Deliverable targets for Aug 9
\begin{todolist}
\item Poster results and discussion
\item Polish and summarize for paper draft
\end{todolist}
\end{itemize}

\ssec{Lab Notes} You write ~\paul{I comment}

05/30/2023 First meeting:
We established E-mail and Slack as the preferred communication methods. 
We agree to meet once a week, 1h time slot to be scheduled.
Explore the flatiron wiki. Use Overleaf for lab notes.

\ssec{List of hot topics in physics}

\begin{itemize}
\item magic-angle graphene
\item TMDC bilayer
\item unconventional superconductivity
\end{itemize}

%\ssec{Results and Discussions}
%\ssec{Conclusion and Outlook}
%\ssec{Acknowledgment}
%We thank the Flatiron Institute Scientific Computing Center for computational resources and technical support. The Flatiron Institute is a division of the Simons Foundation.

\bibliographystyle{apsrev4-1}
\bibliography{ref}

\end{document}
